\documentclass[12pt]{article}
\usepackage{lmodern}
\usepackage[margin=0.8in]{geometry}
\usepackage{amssymb,amsmath}
\usepackage{ifxetex,ifluatex}
\usepackage{fixltx2e} % provides \textsubscript
\ifnum 0\ifxetex 1\fi\ifluatex 1\fi=0 % if pdftex
  \usepackage[T1]{fontenc}
  \usepackage[utf8]{inputenc}
\else % if luatex or xelatex
  \ifxetex
    \usepackage{mathspec}
    \usepackage{xltxtra,xunicode}
  \else
    \usepackage{fontspec}
  \fi
  \defaultfontfeatures{Mapping=tex-text,Scale=MatchLowercase}
  \newcommand{\euro}{€}
\fi
% use upquote if available, for straight quotes in verbatim environments
\IfFileExists{upquote.sty}{\usepackage{upquote}}{}
% use microtype if available
\IfFileExists{microtype.sty}{%
\usepackage{microtype}
\UseMicrotypeSet[protrusion]{basicmath} % disable protrusion for tt fonts
}{}
\ifxetex
  \usepackage[setpagesize=false, % page size defined by xetex
              unicode=false, % unicode breaks when used with xetex
              xetex]{hyperref}
\else
  \usepackage[unicode=true]{hyperref}
\fi
\hypersetup{breaklinks=true,
            bookmarks=true,
            pdfauthor={Brandon LeBeau},
            pdftitle={PSQF 4143: Section 1},
            colorlinks=true,
            citecolor=blue,
            urlcolor=blue,
            linkcolor=magenta,
            pdfborder={0 0 0}}
\urlstyle{same}  % don't use monospace font for urls
\setlength{\parindent}{0pt}
\setlength{\parskip}{6pt plus 2pt minus 1pt}
\setlength{\emergencystretch}{3em}  % prevent overfull lines
\setcounter{secnumdepth}{0}

\title{PSQF 4143: Section 1}
\author{Brandon LeBeau}
\date{}

\begin{document}
\maketitle

\section{Introduction to Statistics}\label{introduction-to-statistics}

\begin{itemize}
\itemsep1pt\parskip0pt\parsep0pt
\item
  What is statistics?

  \begin{itemize}
  \itemsep1pt\parskip0pt\parsep0pt
  \item
    Statistics is the science of gaining information from numerical data
  \end{itemize}
\item
  The root of statistics comes from numerical data

  \begin{itemize}
  \itemsep1pt\parskip0pt\parsep0pt
  \item
    Produce data
  \item
    Organize data
  \item
    Draw conclusions from data
  \end{itemize}
\end{itemize}

\section{Descriptive Statistics}\label{descriptive-statistics}

\begin{itemize}
\itemsep1pt\parskip0pt\parsep0pt
\item
  Descriptive statistics aim to summarize or explore the data

  \begin{itemize}
  \itemsep1pt\parskip0pt\parsep0pt
  \item
    Tell a story of data analysis
  \end{itemize}
\item
  These statistics can take the form of graphics, tables, or single
  numerical summaries
\item
  Examples of this include:

  \begin{itemize}
  \itemsep1pt\parskip0pt\parsep0pt
  \item
    Make summary statements about the data, on average\ldots{}
  \item
    Explore the distribution of the data, where do the bulk of the data
    fall?
  \item
    Explore similarities or differences between groups
  \end{itemize}
\end{itemize}

\section{Inferential Statistics}\label{inferential-statistics}

\begin{itemize}
\itemsep1pt\parskip0pt\parsep0pt
\item
  Inferential statistics aim to quantify uncertainty in the data and use
  probability to understand how likely a hypothesis is.
\item
  Many inferential techniques aim to answer questions that explore the
  relationship between two or more variables.
\item
  Inferential statistics are useful because it is common in an
  experiment to not have data on everything.

  \begin{itemize}
  \itemsep1pt\parskip0pt\parsep0pt
  \item
    Example: Imagine testing for the presence of bacteria at a beach.
  \item
    Example: Imagine exploring the relationship between amount of sleep
    the night before a test and the performance on a test.
  \end{itemize}
\end{itemize}

\section{Data Uncertainty}\label{data-uncertainty}

\begin{itemize}
\itemsep1pt\parskip0pt\parsep0pt
\item
  Data vary
\item
  As a result, conclusions on uncertain data are also uncertain.
\item
  Statistics is a tool to help us quantify the uncertainty and use that
  to make meaningful conclusions.
\item
  Statistics helps us determine if a result is spurious (uncertain) or
  meaningful.
\end{itemize}

\section{Scaffolding for Research}\label{scaffolding-for-research}

\begin{itemize}
\itemsep1pt\parskip0pt\parsep0pt
\item
  Think back to the scientific method from science classes in high
  school.

  \begin{enumerate}
  \def\labelenumi{\arabic{enumi}.}
  \itemsep1pt\parskip0pt\parsep0pt
  \item
    Research Question
  \item
    Statistical Question
  \item
    Data Collection
  \item
    Data Analysis
  \item
    Statistical Conclusion
  \item
    Research Conclusion
  \end{enumerate}
\end{itemize}

\section{Questions for Statistics}\label{questions-for-statistics}

\begin{itemize}
\itemsep1pt\parskip0pt\parsep0pt
\item
  The following are useful questions to always have in mind when
  considering a statistical technique:

  \begin{itemize}
  \itemsep1pt\parskip0pt\parsep0pt
  \item
    What assumptions underlie the technique?
  \item
    When and where is the technique valid?
  \item
    What are the advantages and disadvantages of the technique relative
    to others?
  \item
    What interpretations can be made based on the technique?
  \item
    What are the common problems associated with the technique?
  \end{itemize}
\end{itemize}

\section{Variables vs Constants}\label{variables-vs-constants}

\begin{itemize}
\itemsep1pt\parskip0pt\parsep0pt
\item
  Variables:

  \begin{itemize}
  \itemsep1pt\parskip0pt\parsep0pt
  \item
    vary
  \item
    are commonly represented by symbols, x, y, z
  \item
    need to be measured. Measurement of a variable is the process of
    assigning numbers or labels to characteristics of people, objects,
    or events according to a set of rules.
  \end{itemize}
\item
  Constants:

  \begin{itemize}
  \itemsep1pt\parskip0pt\parsep0pt
  \item
    do not vary
  \item
    are commonly represented by symbols, a, b, c
  \end{itemize}
\end{itemize}

\section{Types of variables}\label{types-of-variables}

\begin{itemize}
\itemsep1pt\parskip0pt\parsep0pt
\item
  Qualitative variables:

  \begin{itemize}
  \itemsep1pt\parskip0pt\parsep0pt
  \item
    represent categories
  \item
    are not numbers
  \item
    but can be represented with numeric symbols
  \item
    can be ordered or unordered
  \end{itemize}
\item
  Quantitative variables:

  \begin{itemize}
  \itemsep1pt\parskip0pt\parsep0pt
  \item
    reflect numeric quantities
  \item
    can be discrete or continuous
  \end{itemize}
\end{itemize}

\section{Levels of Measurement}\label{levels-of-measurement}

\begin{itemize}
\itemsep1pt\parskip0pt\parsep0pt
\item
  Nominal Measurement:

  \begin{itemize}
  \itemsep1pt\parskip0pt\parsep0pt
  \item
    represent mutually exclusive categories
  \item
    are commonly represented with labels
  \item
    can be represented with numeric labels
  \item
    is not meaningful to manipulate mathematically
  \item
    carries no meaningful order
  \end{itemize}
\item
  Ordinal Measurement:

  \begin{itemize}
  \itemsep1pt\parskip0pt\parsep0pt
  \item
    carries all information from nominal measurement
  \item
    represent mutually exclusive categories that are ordered
  \end{itemize}
\item
  Interval Measurement:
\item
  carries all information from nominal and ordinal measurement
\item
  represent equal distance between equivalent gaps
\item
  0 is not meaningful
\item
  can now perform linear transformations meaningfully
\item
  Ratio Measurement:
\item
  carries all information from nominal, ordinal, and interval
  measurement
\item
  but now 0 measn the absence of the phenomenon
\item
  can only be transformed by multiplying
\end{itemize}

\section{Measurement in the Social
Sciences}\label{measurement-in-the-social-sciences}

\begin{itemize}
\itemsep1pt\parskip0pt\parsep0pt
\item
  When measuring variables with an instrument or questionnaire, there is
  commonly a lack of equal intervals or an absolute zero.
\item
  Examples: creativity, problem solving, attitude toward statistics,
  achievement

  \begin{itemize}
  \itemsep1pt\parskip0pt\parsep0pt
  \item
    The distances between adjacent scores based on an instrument are
    likely not equal.
  \item
    For an achievement test, if an individual can not answer any
    questions, that does not mean they have a total absence of
    ability/achievement.
  \end{itemize}
\end{itemize}

\section{Measurement of Ordered
Variables}\label{measurement-of-ordered-variables}

\begin{itemize}
\itemsep1pt\parskip0pt\parsep0pt
\item
  The purpose of measurement is to differentiate among individuals or
  objects on a variable.
\item
  To do this, rules must be established to assign values to a specific
  variable.
\item
  When the rules for a measurement process are not widely accepted, we
  must adopt a two-step measurement process.
\end{itemize}

\section{Two Step Measurement
Process}\label{two-step-measurement-process}

\begin{enumerate}
\def\labelenumi{\arabic{enumi}.}
\itemsep1pt\parskip0pt\parsep0pt
\item
  Develop a \emph{conceptual} definition.

  \begin{itemize}
  \itemsep1pt\parskip0pt\parsep0pt
  \item
    This is abstract or a theoretical definition
  \end{itemize}
\item
  Develop an \emph{operational} definition
\end{enumerate}

\begin{itemize}
\itemsep1pt\parskip0pt\parsep0pt
\item
  More concrete
\item
  Based on the conceptual definition
\item
  Commonly defines rules to allow for the measurement of the phenomenon.
\end{itemize}

\section{Two Step Measurement Process
Example}\label{two-step-measurement-process-example}

\begin{itemize}
\itemsep1pt\parskip0pt\parsep0pt
\item
  Consider the variable motivation
\item
  Conceptual Definition

  \begin{itemize}
  \itemsep1pt\parskip0pt\parsep0pt
  \item
    The general desire or willingness of someone to do something
  \end{itemize}
\item
  Operational Definition

  \begin{itemize}
  \itemsep1pt\parskip0pt\parsep0pt
  \item
    Give a list of 50 statements of tasks.
  \item
    Then ask each individual to indicate the extent they enjoy doing
    each task - from strongly do not enjoy to strongly enjoy.
  \end{itemize}
\end{itemize}

\section{Reliability}\label{reliability}

\begin{itemize}
\itemsep1pt\parskip0pt\parsep0pt
\item
  A measurement process is reliable if \emph{repeated measurements} on
  the \emph{same individual} give the same (or similar) results.
\item
  Examples:

  \begin{itemize}
  \itemsep1pt\parskip0pt\parsep0pt
  \item
    If a second set of 50 similar tasks were given to the same
    individuals, would individuals order the tasks similarly?
  \item
    If the same set of 50 tasks were given to the same individuals,
    would the students be order the tasks similarly?
  \end{itemize}
\end{itemize}

\section{Validity}\label{validity}

\begin{itemize}
\itemsep1pt\parskip0pt\parsep0pt
\item
  A variable is a valid measure of a property if it is relevant or
  appropriate as a representation of that property.
\item
  Is the ordering of individuals based on a particular operational
  definition the same as (or similar to) the ordering based on a
  different operational definition?
\item
  A common question to test validity is to ask ourselves whether the
  measure reflects the construct of interest.

  \begin{itemize}
  \itemsep1pt\parskip0pt\parsep0pt
  \item
    More simply, are we measuring what we want or something else?
  \end{itemize}
\end{itemize}

\section{Real Limits}\label{real-limits}

\begin{itemize}
\itemsep1pt\parskip0pt\parsep0pt
\item
  The real limits of a score extend from one-half of the smallest unit
  of measurement below the value of the score to one-half unit above
\item
  Examples:

  \begin{itemize}
  \itemsep1pt\parskip0pt\parsep0pt
  \item
    Consider a distance of 85 inches, measured to the nearest inch (the
    smallest unit of measurement is one inch).

    \begin{itemize}
    \itemsep1pt\parskip0pt\parsep0pt
    \item
      The real limits of 85 extend from 84.5 to 85.5.
    \item
      As such, a score of 85 inches actually represents all values
      between 84.5 and 85.5.
    \end{itemize}
  \item
    Suppose we measure length to the nearest tenth of an inch

    \begin{itemize}
    \itemsep1pt\parskip0pt\parsep0pt
    \item
      A measurement of 2.3 inches - what are the real limits?
    \item
      2.25 to 2.35 inches.
    \end{itemize}
  \end{itemize}
\item
  Unfortunately there are exceptions.

  \begin{itemize}
  \itemsep1pt\parskip0pt\parsep0pt
  \item
    Age is one example
  \end{itemize}
\item
  No score can ever fall right on a real limit, because we calculate
  real limits by taking half of the smallest unit of measurement.
\end{itemize}

\section{Real Limits 2}\label{real-limits-2}

\begin{itemize}
\itemsep1pt\parskip0pt\parsep0pt
\item
  Example:

  \begin{itemize}
  \itemsep1pt\parskip0pt\parsep0pt
  \item
    Consider grade point averages in the interval from 2.60 to 2.79.

    \begin{itemize}
    \itemsep1pt\parskip0pt\parsep0pt
    \item
      The smallest unit of measurement is one-hundredth.
    \item
      The lower and upper real limits of the interval are 2.595 and
      2.795 respectively.
    \end{itemize}
  \end{itemize}
\end{itemize}

\section{Population vs Sample}\label{population-vs-sample}

\begin{itemize}
\itemsep1pt\parskip0pt\parsep0pt
\item
  Population:

  \begin{itemize}
  \itemsep1pt\parskip0pt\parsep0pt
  \item
    The complete set of observations about which a researcher wishes to
    draw a conclusion.
  \item
    A descriptive index of a population is referred to as a
    \emph{parameter}.
  \item
    Population Parameter
  \end{itemize}
\item
  Sample:

  \begin{itemize}
  \itemsep1pt\parskip0pt\parsep0pt
  \item
    Part of the population about which a researcher wishes to draw a
    conclusion.
  \item
    A descriptive index of a sample is referred to as a
    \emph{statistic}.
  \item
    Sample Statistic
  \end{itemize}
\end{itemize}

\section{Random Sample}\label{random-sample}

\begin{itemize}
\itemsep1pt\parskip0pt\parsep0pt
\item
  A random sample is a sample obtained such that each individual has an
  equal chance of being selected at any stage of the sampling process.
\item
  Also, each possible sample of the same size has an equal probability
  of being selected from the population.
\item
  A population is defined by the interest of the investigator.
\end{itemize}

\section{Inferential Statistics}\label{inferential-statistics-1}

\begin{itemize}
\itemsep1pt\parskip0pt\parsep0pt
\item
  With inferential statistics, we are interested in estimating
  population parameters from sample statistics.
\item
  An essential assumption in inferential statistics is that samples are
  drawn randomly from a particular population.
\item
  If several random samples of the same size are drawn from the same
  population, the samples will most likely differ, and therefore their
  characteristics (statistics) will vary from sample to sample.
\item
  The variation is due to chance variation associataed with random
  sampling.

  \begin{itemize}
  \itemsep1pt\parskip0pt\parsep0pt
  \item
    This is commonly called sampling error.
  \end{itemize}
\item
  The larger the random sample from the population, the less the
  variation between samples.
\item
  In general larger random samples will provide a more precise estimate
  of what is true about a population.
\end{itemize}

\end{document}
