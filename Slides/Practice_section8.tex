\documentclass[11pt]{article}
\usepackage[margin = 1in]{geometry}
\usepackage[T1]{fontenc}
\usepackage{lmodern}
\usepackage{amssymb,amsmath}
\usepackage{ifxetex,ifluatex}
\usepackage{fixltx2e} % provides \textsubscript
% use upquote if available, for straight quotes in verbatim environments
\IfFileExists{upquote.sty}{\usepackage{upquote}}{}
\ifnum 0\ifxetex 1\fi\ifluatex 1\fi=0 % if pdftex
  \usepackage[utf8]{inputenc}
\else % if luatex or xelatex
  \ifxetex
    \usepackage{mathspec}
    \usepackage{xltxtra,xunicode}
  \else
    \usepackage{fontspec}
  \fi
  \defaultfontfeatures{Mapping=tex-text,Scale=MatchLowercase}
  \newcommand{\euro}{€}
\fi
% use microtype if available
\IfFileExists{microtype.sty}{\usepackage{microtype}}{}
\usepackage{graphicx}
\usepackage{float}
% Redefine \includegraphics so that, unless explicit options are
% given, the image width will not exceed the width of the page.
% Images get their normal width if they fit onto the page, but
% are scaled down if they would overflow the margins.
\makeatletter
\def\ScaleIfNeeded{%
  \ifdim\Gin@nat@width>\linewidth
    \linewidth
  \else
    \Gin@nat@width
  \fi
}
\makeatother
\let\Oldincludegraphics\includegraphics
{%
 \catcode`\@=11\relax%
 \gdef\includegraphics{\@ifnextchar[{\Oldincludegraphics}{\Oldincludegraphics[width=\ScaleIfNeeded]}}%
}%
\ifxetex
  \usepackage[setpagesize=false, % page size defined by xetex
              unicode=false, % unicode breaks when used with xetex
              xetex]{hyperref}
\else
  \usepackage[unicode=true]{hyperref}
\fi
\hypersetup{breaklinks=true,
            bookmarks=true,
            pdfauthor={Brandon LeBeau},
            pdftitle={ESRM 5393: Inference Introduction},
            colorlinks=true,
            citecolor=blue,
            urlcolor=blue,
            linkcolor=magenta,
            pdfborder={0 0 0}}
\urlstyle{same}  % don't use monospace font for urls
\setlength{\parindent}{0pt}
\setlength{\parskip}{6pt plus 2pt minus 1pt}
\setlength{\emergencystretch}{3em}  % prevent overfull lines
\setcounter{secnumdepth}{0}

\begin{document}

\section{Central Limit Theorem}\label{central-limit-theorem}

\begin{itemize}
\itemsep1pt\parskip0pt\parsep0pt
\item
  For a randomly selected sample of size $n$ with a mean $\mu$ and a
  standard deviation $\sigma$, the following is true:

  \begin{enumerate}
  \def\labelenumi{\arabic{enumi}.}
  \itemsep1pt\parskip0pt\parsep0pt
  \item
    The distribution of sample means $\bar{X}$ is approximately normal,
    regardless of the population distribution. For markedly non-normal populations, the approximation is good enough when the sample size is 25 or more.
  \item
    The mean of the distribution of sample means is equal to the mean of
    the population distribution, $\mu_{\bar{x}} = \mu$.
  \item
    The standard deviation of the distribution of sample means is equal
    to: $\sigma_{\bar{X}} = \frac{\sigma}{\sqrt{n}}$
  \end{enumerate}
\end{itemize}

\section{Examples}\label{examples}

\begin{itemize}
\itemsep1pt\parskip0pt\parsep0pt
\item
  Given a normally distributed population with \(\mu_{X} = 70\) and
  \(\sigma_{X} = 20\); that is \(X \sim N(70, 20)\)
\item
  Assume that we take a random sample of size \(n = 25\)
\item Use with examples 1 - 4
\end{itemize}

\section{Example 1}\label{example-1}

\begin{itemize}
\itemsep1pt\parskip0pt\parsep0pt
\item
  What is the probability of obtaining a random sample with a mean of 80
  or higher? \[ Pr (X \geq 80 | X \sim N(70, 20)) \]
  \[ Pr (\bar{X} \geq 80 | \bar{X} \sim N(70, 4)) \]
\end{itemize}

\section{Example 2}\label{example-2}

\begin{itemize}
\itemsep1pt\parskip0pt\parsep0pt
\item
  What is the probability of obtaining a random sample with a mean that
  differs from the population mean by more than 10 points?
  \[ Pr (X \geq 80 or X \leq 60 | X \sim N(70, 20)) \]
  \[ Pr (\bar{X} \geq 80 or \bar{X} \leq 60 | \bar{X} \sim N(70, 4)) \]
  \[ Pr (|\bar{X} - \mu_{X}| \geq 10 | \bar{X} \sim N(70, 4)) \]
\end{itemize}

\section{Example 3}\label{example-3}

\begin{itemize}
\itemsep1pt\parskip0pt\parsep0pt
\item
  What sample mean has a value such that the probability of obtaining
  one at least that high in random sampling is .05?
\item
  Find \(\bar{X}_{o}\) such that:
  \[ Pr (\bar{X} \geq \bar{X}_{o} | \bar{X} \sim N(70, 4)) = 0.05 \]
\end{itemize}

\section{Example 4}\label{example-4}

\begin{itemize}
\itemsep1pt\parskip0pt\parsep0pt
\item
  Within what limits would the central 95\% of the sample means fall?
\item
  Find \(\bar{X}_{1}\) and \(\bar{X}_{2}\) such that:
  \[ Pr (\bar{X}_{1} \leq \bar{X} \leq \bar{X}_{2} | \bar{X} \sim N(70, 4)) = 0.95 \]
\end{itemize}

\section{Example 5}

Suppose we are interested in knowing on average how early students show up prior to kickoff for a home UI football game. We know the population has $\mu = 15$, $\sigma = 20$, and the population is severely positively skewed. Use this information to answer the following questions:
\begin{itemize}
\item We take a random sample of 10 students and get a sample mean of $\bar{X} = 45$. Can we safely use the central limit theorem here? 
\item We take a random sample of 50 students where $\bar{X} = 19$. What is the probability of getting a sample mean greater or equal to this value? How many students would we expect to show up 19 minutes or earlier (assume 10000 students attend the game).
\item What are the limits in which the central 95\% of the sample means will fall when $n = 25$ and when $n = 100$? How do these compare? Why is this pattern occurring?
\item Given a random sample of 25 students ($n = 25$), how likely is it for those 25 students to on average show up at kickoff or after?
\end{itemize}


\end{document}