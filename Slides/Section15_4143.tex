\documentclass[12pt]{article}
\usepackage[margin=0.75in]{geometry}
\usepackage{float}
\usepackage{multicol}
\usepackage{lmodern}
\usepackage{amssymb,amsmath}
\usepackage{ifxetex,ifluatex}
\usepackage{fixltx2e} % provides \textsubscript
\ifnum 0\ifxetex 1\fi\ifluatex 1\fi=0 % if pdftex
  \usepackage[T1]{fontenc}
  \usepackage[utf8]{inputenc}
\else % if luatex or xelatex
  \ifxetex
    \usepackage{mathspec}
    \usepackage{xltxtra,xunicode}
  \else
    \usepackage{fontspec}
  \fi
  \defaultfontfeatures{Mapping=tex-text,Scale=MatchLowercase}
  \newcommand{\euro}{€}
\fi
% use upquote if available, for straight quotes in verbatim environments
\IfFileExists{upquote.sty}{\usepackage{upquote}}{}
% use microtype if available
\IfFileExists{microtype.sty}{%
\usepackage{microtype}
\UseMicrotypeSet[protrusion]{basicmath} % disable protrusion for tt fonts
}{}
\ifxetex
  \usepackage[setpagesize=false, % page size defined by xetex
              unicode=false, % unicode breaks when used with xetex
              xetex]{hyperref}
\else
  \usepackage[unicode=true]{hyperref}
\fi
\hypersetup{breaklinks=true,
            bookmarks=true,
            pdfauthor={Brandon LeBeau},
            pdftitle={PSQF 4143: Section 15},
            colorlinks=true,
            citecolor=blue,
            urlcolor=blue,
            linkcolor=magenta,
            pdfborder={0 0 0}}
\urlstyle{same}  % don't use monospace font for urls
\setlength{\parindent}{0pt}
\setlength{\parskip}{6pt plus 2pt minus 1pt}
\setlength{\emergencystretch}{3em}  % prevent overfull lines
\setcounter{secnumdepth}{0}

\title{PSQF 4143: Section 15}
\author{Brandon LeBeau}
\date{}

\begin{document}
\maketitle

\section{Hypothesis Tests for
Correlations}\label{hypothesis-tests-for-correlations}

\begin{itemize}
\itemsep1pt\parskip0pt\parsep0pt
\item
  Many researchers may be interested in testing if the correlation
  differs from 0.
\item
  The hypotheses would be:

  \begin{itemize}
  \itemsep1pt\parskip0pt\parsep0pt
  \item
    \(H_{0}: \rho = 0\)
  \item
    \(H_{1}: \rho \neq 0\)
  \item
    Can also do one-sided hypotheses here.
  \end{itemize}
\end{itemize}

\section{Test Statistic}\label{test-statistic}

\[ TS = \frac{r\sqrt{n - 2}}{\sqrt{1 - r^2}} \sim t_{df} \]

\begin{itemize}
\itemsep1pt\parskip0pt\parsep0pt
\item
  \(df = n - 2\)
\item
  \(n\) denotes the number of pairs of observations
\end{itemize}

\section{Example}\label{example}

\begin{itemize}
\itemsep1pt\parskip0pt\parsep0pt
\item
  A researcher wants to determine whether \(r = 0.66\) is significantly
  greater than 0.
\item
  The sample size was 50 (50 observations on X and Y scores)
\item
  Let's use \(\alpha = .01\)
\end{itemize}

\section{Testing a constant other than
0}\label{testing-a-constant-other-than-0}

\begin{itemize}
\itemsep1pt\parskip0pt\parsep0pt
\item
  The sampling distribution is distributed as a \(t_{df}\) when testing
  against 0.
\item
  However, what if we wanted to test the following hypotheses?

  \begin{itemize}
  \itemsep1pt\parskip0pt\parsep0pt
  \item
    \(H_{0}: \rho = 0.50\)
  \item
    \(H_{1}: \rho > 0.50\)
  \end{itemize}
\item
  Since the correlation can only go as large as +1 (or as small as -1),
  testing constants other than 0 can lead to skewed sampling
  distributions.
\item
  Fortunately, a procedure developed by R.A. Fisher can be used in these
  situations.
\end{itemize}

\section{Fisher's r to z
Transformation}\label{fishers-r-to-z-transformation}

\begin{itemize}
\itemsep1pt\parskip0pt\parsep0pt
\item
  We will use the table D.7 of the course packet (page 230) to convert
  correlations (\(r\)) to the \(z'\) metric.

  \begin{itemize}
  \itemsep1pt\parskip0pt\parsep0pt
  \item
    Note, that for negative correlations, just make the \(z'\) value
    negative.
  \end{itemize}
\item
  The sampling distribution will now be normally distributed.
\end{itemize}

\[ TS = \frac{z' - z'_{0}}{\frac{1}{\sqrt{n - 3}}} \sim Z \]
\[ \hat{\sigma}_{z'} = \frac{1}{\sqrt{n - 3}} \]

\section{Example}\label{example-1}

\begin{itemize}
\itemsep1pt\parskip0pt\parsep0pt
\item
  A researcher wants to determine whether \(r = 0.66\) is significantly
  greater than 0.50.
\item
  The sample size was 50 (50 observations on X and Y scores)
\item
  Let's use \(\alpha = .05\)
\end{itemize}

\section{Assumptions}\label{assumptions}

\begin{enumerate}
\def\labelenumi{\arabic{enumi}.}
\itemsep1pt\parskip0pt\parsep0pt
\item
  \(r\) computed from a random sample
\item
  The population is bivariate normal
\item
  \(n > 10\)
\item
  \(\rho\) is not too close to 1 or -1
\end{enumerate}

\section{Confidence interval for a
correlation}\label{confidence-interval-for-a-correlation}

\begin{itemize}
\itemsep1pt\parskip0pt\parsep0pt
\item
  A two-sided confidence interval for \(z'_{pop}\) is given by:
\end{itemize}

\[ z' \pm z_{crit} \frac{1}{\sqrt{n - 3}} \]

\begin{itemize}
\itemsep1pt\parskip0pt\parsep0pt
\item
  The confidence limits calculated above are in the \(z'\) metric. After
  calculation, we will backtransform them into the \(r\) metric using
  table D.7 from the coursepacket.
\end{itemize}

\section{Example}\label{example-2}

\begin{itemize}
\itemsep1pt\parskip0pt\parsep0pt
\item
  Find a 95\% confidence interval for \(\rho\), given that a sample of
  \(n = 50\) and \(r = .66\).
\end{itemize}

\end{document}
